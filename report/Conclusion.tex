\section{Conclusion}
Given the environmental imperative to reduce greenhouse gas emissions in all aspects of life, we take a closer look at the policy of daylight saving time (\ac{DST}) through which people's clock-based behaviour is more aligned with sunshine. In an electricity network with a significant percentage of energy produced by solar, a synchronisation between this production and the demand might result in a reduction in overall greenhouse gas emissions, when controlling for all possible confounders. 
We analyse the effect of DST on $CO_2$ emissions and electricity in Australia, taking advantage of the heterogeneous adoption of DST in the nation for a time span of fourteen years from 2009 until 2023 with two \ac{DST}-transitions per year. Using half-hourly panel data on electricity and $CO_2$ production per region, we apply a \ac{DDD} treatment effect model. We use the framework to possibly identify significant differences during \ac{DST} between the treatment and control regions, and using midday loads, which are not affected by the demand shift, to control for inter-day shocks. Our results do not suggest that a significant effect of DST exists on $CO_2$ equivalent emissions nor on electricity production. The interpretation of our results remains mostly unaffected by multiple robustness checks. Compared to previous literature, our results align with the findings of \textcites{kellogg_daylight_2008}, who also do not find a significant aggregate effect but instead an intra-day shift. 
Thus, our results could also bring a new impetus to the ongoing policy-discussions of \ac{DST}, since no strong evidence is found for the original energy-saving argument of the policy. 