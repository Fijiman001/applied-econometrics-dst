\section{Methodology and Model}

We analyse $CO_2$ emissions from the electricity grid before and after the exogenous \ac{DST} change using a \acf{DiD} framework. To compare our results to the literature we additionally run an equivalent specification looking at electricity consumption in $kWh$. As shown in Figure \ref{fig:map}, we specify Queensland, which does not use \ac{DST}, as our control and the other states in the \ac{NEM} as the treatment. To get plausible coefficients from the \ac{DiD} regression, we first verify the common prior trend assumption, using an event-study graph.
For this, we use the \ac{DST} time shift in the respective states as the treatment, with emissions and electricity consumption as the outcomes, as shown in Figure \ref{fig:intraday co2}. We include further controls and entity and time-fixed effects for every state and date to isolate the effect of \ac{DST} from possible confounders. \\ 
Further controls include average sunlight irradiance, average wind speed, maximum daily temperature, weekends and public holidays. \ac{DST} coincides generally with longer daylight hours, and thus greater solar power generation, which displaces fossil fuel generation and can potentially reduce emissions. To control for this, average daily sunlight irradiance (adjusting for cloud cover and the Earth's tilt) is used as a control. Similarly, average wind speed is used as a control for wind generation. Maximum daily temperature is used as a control to explain heating and cooling loads.\footnote{Due to thermal inertia, instantaneous temperature would be a less effective control} Weather data is aggregated at a daily level because intra-day data (especially sun intensity) would be a collider. %Temperature data was taken from capital cities, since most thermal load is consumed in capital cities. Wind and sunlight data was taken from region midpoints, as representative values for generation which is typically dispersed across the region. 
Controls were added for weekend and public holidays as they tend to reduce electricity demand significantly.

To correct for a potentially missing common prior trend and improved interpretability we follow \textcite{kellogg_daylight_2008} and additionally implement a \ac{DDD} design, first performed by \textcite{gruber_incidence_1994}, and the associated event study. The \ac{DiD} framework allows us to identify the average differences between the treatment and control states. However, it does not control for state-specific demand shifts. To do so, we make use of the fact that electricity demand during the midday period does not see a shift from DST, compared to morning and evening peaks.\footnote{compare \textcite{kellogg_daylight_2008}}


\subsection{Difference-in-differences (DiD) Estimation}
Following \textcites{callaway_difference--differences_2021, goodman-bacon_difference--differences_2021}, Equation \ref{eq:DD} shows the \ac{DiD} regression we implement.
\begin{equation}
    \left(\frac{CO_2}{Population}\right)_{r,t}
 = \beta_0 + \beta_1*Treatment_{r} + \beta_2Post_{t} + \beta_3(Treatment \times Post)_{r,t} + \beta_4 Controls_{r,t}
     + \epsilon_{r,t}
     \label{eq:DD}
\end{equation}

\textit{Treatment} is a binary variable, which is 1 if a region has in general implemented a policy for \ac{DST} (i.e. all studied regions except Queensland). \textit{Post} is 1 during each period in which \ac{DST} is active (October to March). Errors are clustered by region to mitigate the impact of serial correlation. Furthermore, the data is weighted by population. We also run the same \ac{DiD} specification with $kWh$ per capita as an alternative outcome.
To examine the common prior trend assumption for both emissions and electricity consumption, equation \ref{eq:ES-DD} shows the event study performed using the Stata package provided by \textcite{clarke_implementing_2021}.
\begin{equation}
    y_{r,t} = \sum_{j=2}^{J} \beta_{-j} \times (Lead_j)_{r,t} + \sum_{k=0}^K \beta_{k} \times (Lag_k)_{r,t} + \mu_r + \lambda_t + X^{'}_{r,t} + \epsilon_{r,t}
    \label{eq:ES-DD}
\end{equation}
where $y_{r,t}$ is $CO_2$ or electricity consumption per capita for region $r$ and time period $t$ respectively. The $\beta$ coefficients represent our event study estimates for the lead and lag effects of DST respectively, based on the time to treatment $t$. In our case, this is the number of days into DST, being negative when DST is not active, 0 on the days when both forward and backward DST transitions occur, and positive when DST is active. $\mu_r$ and $\lambda_t$ are entity and time-fixed effects by region and day respectively. $X_{r,t}$ represent a vector of controls per region and day and their respective coefficients.  

\subsection{Difference-in-difference-in-difference (DDD) Estimation}

The DDD design allows us to correct for unobserved factors in the \ac{DiD}-framework affecting the control and treatment groups differently. To establish the \ac{DDD}, we normalise by emissions and electricity demand during 12:00-14:30. Midday emissions and electricity demand respectively will be the largely unaffected by the time shift from DST, because the sun is at its highest point. 
\begin{align}
    \label{eq:DDD}
    \left(\frac{CO_2}{Population}\right)_{r,t,m} &= \beta_0 + \beta_1Treatment_{r} + \beta_2Post_{t} + \beta_3NotMidday_{m}   \\
    & +\beta_4(Treatment \times Post)_{r,t} + 
    \beta_5(Treatment \times NotMidday)_{r,m} \nonumber \\ 
    & +\beta_6(Post \times NotMidday)_{t,m} + \beta_7 (Treatment \times Post \times NotMidday)_{r,t,m} \nonumber \\ 
    &+ \beta_8 Controls_{r,t}  + \epsilon_{r,t,m}
    \nonumber 
\end{align}
Adding this third difference, we run our DDD regressions (Equation \ref{eq:DDD}) with the same data, the same additional outcome of $kWh$ $p.c$, and clustering by region and weighting by population. The variable of the third difference (\textit{NotMidday}) is 0 for the half hours between 12:00 and 14:30 (local time), and 1 otherwise, with the subscript $m$ specifying whether the observation takes place during the midday period.
To check the prior common trends assumption we make, we again apply the event study design shown in Equation \ref{eq:ES-DD}, adjusting our outcome variables by the midday values for $CO_2$ and electricity production respectively. This is equivalent to taking ratios of the outcome variable, to allow us to create a quasi-equivalent event study design for the \ac{DDD} regression we perform, as specified by \textcite{olden_triple_2022}. The main coefficient of interest is $\beta_7$ that captures the effect of not being in the period between 12 and 14:30, in a treatment region, while \ac{DST} is active.

