\section{Literature Review}
There has been extensive research into the social and individual side effects of \ac{DST}, such as increases in heart attacks, increases in criminal sentencing severity, decreases in crime and decreases in traffic accidents \parencite{heart_attacks, sleepy_punishers, doleac_crime, bunnings_traffic}.
However, fewer authors have researched the effect \ac{DST} has on $CO_2$ emissions and electricity consumption. \\
% DOE
% Choi
% Guven
% Hill et al., 2010
% Hancevic and Margulis, 2016
% Kandel and Sheridan, 2007; Momani et al., 2009; Rivers, 2016; Sexton and Beatty, 2014
\textcite{aries_effect_2008} give a comprehensive overview of the literature concerning the effect of \ac{DST} on energy usage for lighting, the argument often cited for the adoption of this policy. They find divided literature on this topic, presenting evidence for and against a decrease in energy usage, concluding that the research in this debate is ``limited, incomplete, or contradictory". Nonetheless, they identify key variables which might affect the results of studies, namely geographical and climatological factors, as these play an important role in the energy consumption patterns of households.\footnote{e.g. in Australia, cooling has a bigger impact on electricity demand than heating} Moreover, they also uncover evidence of a shift in demand within the day, particularly from the morning to the evening hours. This shift occurs because consumers can significantly decrease their lighting usage during the morning, given that the sun is already shining, but are more likely to increase it earlier in the evening, as the sun is setting earlier as well.
\textcite{kotchen} examine inter-day residential energy consumption patterns in Indiana (USA), where they find that \ac{DST} correlates with an overall energy demand increase. They attribute this to a decrease in energy utilization for lighting, which is offset by heightened demand for heating and cooling purposes. The study does not dissect the consumption variations throughout the day but assumes a constant fuel mix and exclusively concentrates on household energy consumption. Consequently, it is hard to infer general results from their findings.
\textcite{kellogg_daylight_2008} analyse the effect  an extension of \ac{DST} for the Sydney 2000 Olympics had on electricity consumption. The authors also detect an intra-day shift in demand and apply a \ac{DDD} framework, in which they assume that the midday load is not affected by \ac{DST} as the third difference, to control for unobservables and shocks. The authors find that the intra-day shift does not affect the overall load in Australia, but only its timing. The study did consider the effect this shift could have on emissions, hypothesizing, that the emissions of generation units serving the peak loads can slightly effect the overall daily emissions.
%did not consider the effect this shift could have on the emissions of the energy sector. 
\textcite{turkey} found a negligible effect on greenhouse gas emissions from a \ac{DiD} analysis of Turkish electricity consumption and demonstrated that an intra-day change in load shape due to \ac{DST} caused a decrease in emissions as a result.\footnote{although in a grid with a high prevalence of hydropower and negligible solar power}

Our research will add a unique contribution to this literature, by making use of the effect on \ac{DST} on an intra-day demand shift, identified in prior research, to examine the emissions within the energy sector. Having access to more recent and detailed data on the Australian intraday fuel mix allows us to use the extensive use of renewable energy sources to investigate the potential impact of aligning electricity demand with sunshine.



%The unique contributions of our research will be focused on how the intra-day time shift in demand from \ac{DST} results in a change in emissions, by altering the fuel mix in a grid with cyclical power supply by photovoltaic systems, using recent data.

% unique contribution:
% more recent
% Australia
% intraday demand
% intraday fuel mix
% solar











