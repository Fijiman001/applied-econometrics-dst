
\section{Introduction}
%We propose researching the question: "". Given the recent trend in ... given then recent debates surrounding ... researching ...
\ac{DST} describes the practice of advancing clocks by one hour during the summer months of the year. For people whose daily schedule is based on clock time, this means that the sun appears to rise and set one hour earlier and they experience an additional hour of sunlight each day, as shown in Figure \ref{fig:sunrise plot}. Today approximately 70 countries have adopted \ac{DST} as a policy \parencite{prerau_book}. Since \ac{DST} was first introduced, the electricity industry has changed drastically. Solar generation is becoming more and more commonplace and inefficient compact fluorescent bulbs have been replaced with far more efficient LEDs. Today, lighting makes up only 15\% of the overall electrical load worldwide \parencite{ec_lighting}. 
Despite being widely adopted over the last century, there is only moderate evidence to support the original energy-saving motivation for \ac{DST} \parencite{prerau_book}. Most literature suggests that \ac{DST} merely shifts load from the evening to the morning, causing a negligible net reduction in electricity consumption, or even a slight increase \parencite{kellogg_daylight_2008, aries_effect_2008, guven}.
In terms of emissions (and cost reduction), shifting \textit{when} consumers consume electricity can be as impactful as reducing the volume consumed \parencite{holland_is_2008}.

We propose that \ac{DST} can act as a synchroniser between electricity demand and emission-free solar power generation. Our study adds to the existing literature by analysing the extent to which DST can change daily $CO_2$ emissions by aligning demand peaks with the availability of solar power. We base our analysis on recent data from Australia, whose grid has substantial solar capacity (34\%, according to the Australian Energy Regulator (2023)\nocite{state_of_the_market}). We make use of the fact that Australian regions had a heterogenous take up of \ac{DST} and that the observed demand shift only affects the morning and evening peaks but does not structurally change consumption behaviour around midday. We control for unobserved state-specific shocks by applying a \ac{DDD} framework. This strategy allows for the necessary identifying assumptions and is more fitting than an \ac{DiD} model. Our results show that \ac{DST} has a positive yet insignificant effect on greenhouse gas emissions, as well as on energy consumption, which is in line with existing literature \parencite{kellogg_daylight_2008}. The result is slightly larger, albeit still insignificant, when looking at the last five years, a period, in which the overall renewable energy production increased. This might be because of the increased uptake of solar over these years.