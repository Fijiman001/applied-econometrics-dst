
\section{Results}
\subsection{Difference-in-Differences (DiD) Results}

The event study plot for the \ac{DiD} design is shown in Figure \ref{fig:dd event study}. As anticipated, the common prior trend does not hold for this specification, confirming our reasoning in running the \ac{DDD} model.
Table \ref{Main-DiD-Results} shows our estimation results but no causal interpretation can occur due to the common prior trend assumption not holding. Looking at the results for emissions, we observe that adding our control variables changes the significance of the \ac{DiD}-coefficient to the point that it is no longer significant at the 5\% level whilst the sign of the coefficient remains negative. We additionally see that the absolute size of the coefficient and therefore the estimated effect of DST on emissions decreases as we add our controls. We observe that only our controls for weekends and public holidays are significant, whilst temperature, solar exposure and wind are insignificant. This can be explained by the fact, that the variation between working and non-working days is bigger than possible variations in the other confounders. 
%Nevertheless, all the coefficients of the controls used have the expected sign/direction, as . 
The results for electricity volume generally do not differ greatly in their interpretation compared to emissions.

\subsection{Difference-in-Difference-in-Differences (DDD) Results}

After controlling for the respective midday values with an additional difference, the event study in Figure \ref{fig:ddd event study} shows that the common prior trends assumption now holds for both electricity consumption and emissions. No clear effect of DST on the respective outcomes can be seen from our event study graphs as our lag coefficients remain centered around $0$.

Tables \ref{Main-DDD-Results} and \ref{DDD-WO-Controls-Results} show the results for the \ac{DDD} regression. 
Compared to our \ac{DiD} design, the sign of our \ac{DDD} coefficient is reversed and positive for both outcomes (0.0163 Kg $CO_2$ and 0.0185 kWh respectively), albeit insignificant. Including our controls does not change this result. Weekend and public holidays remain significant, whilst our other controls remain insignificant. A possible reason for this is, that by controlling for the midday values, we control for most of the between-day and region variations that would be explained by these controls. 

\subsection{Robustness Checks and Limitations}

We perform multiple robustness checks. Tasmania is a region which differs greatly from the control region. It has far lower emissions, mainly hydropower, no solar, colder temperatures and a larger weather difference between summer and winter.
However, given that Tasmania's population is notably small, excluding Tasmania from the analysis had minimal impact on the results (\ref{fig:ddd event study w/o Tas}), due to the weighting of the results according to population already carried out.
After dropping some zero-emission periods (that are only present for Tasmania), the regressions can be performed using the natural logarithm of the dependent variables. As shown in Table \ref{DDD-ln-Results}, there are no major differences in significance, whilst our main coefficients of interest now represent a $1.86\%$ and $1.53\%$ increase in emissions and electricity production respectively. Solely taking the last 5 years of data as a robustness check increased the size of our coefficients of interest, whilst still being insignificant to the 5\% level (see Figure \ref{fig:ddd event study - Last 5} and Table \ref{DDD-Last-5-Results}). The increase is in line with our hypothesis that the larger percentage of solar energy in Australia's electricity grid in recent years has an emissions-reducing effect on our results. Additionally, we separated our regressions and event studies by DST transition direction, as shown in Figure \ref{fig:ddd event study co2 start stop}. Whilst different coefficients are observed between shifting the time forwards and shifting backwards in the event study graphs and the \ac{DDD} results, our results remain insignificant.
Finally, the 12:00-14:30 midday control period was initially calculated using local time, as per \cite{kellogg_daylight_2008}. Using standard Queensland time for all regions instead did not meaningfully change the result.

Although our results hold to multiple robustness checks, there are still limitations to our study. Firstly, although we do cluster our standard errors on the region level to account for serial-correlation, this approach is somewhat flawed since the number of clusters used is very small. 
Furthermore, our data is for the overall aggregated grid, lacking more granular household-level demand data, that might give clearer insights into the causal effects of \ac{DST}. An interesting future study would be to do a regression discontinuity design, comparing households in treated and untreated regions. Furthermore, the closer location of Queensland to the equator directly affects the effectiveness of \ac{DST} and consequently its non-reintroduction. Although we control for multiple factors, we can not be certain to isolate the true causal effect of \ac{DST} on emissions. 