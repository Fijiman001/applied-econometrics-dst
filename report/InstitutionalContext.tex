\section{Institutional Context and Data}
\subsection{Institutional Context}
\ac{DST} was first introduced country-wide in the German Empire and Austria in 1916 in order to support the war effort by minimising the use of fuel for artificial lighting \parencite{Reichsgesetzblatt}. Due to its energy-saving nature, the policy gained popularity among numerous countries, most notably almost all European countries, the US and Russia. \ac{DST} has seen many revisions and adjustments throughout the years \parencite{prerau_book}. It has been active nationwide in the US since 1966 \parencite{Uniform} and in most parts of Europe since the 1970s energy crisis \parencite{pearce_great_2017}.  
However, nowadays with the widespread adoption of highly efficient LED lightbulbs, the original energy-saving argument of \ac{DST} no longer applies \parencite{aries_effect_2008}.
Still, the policy remains active in approximately 70 countries \parencite{prerau_book}, excluding those near the equator where the minimal variation in sunrise and sunset times renders its implementation unnecessary. As of today, policymakers in many of those regions using \ac{DST} are debating the abandonment of the policy. The European Union, Mexico, and the United States of America have considered discontinuing DST due to anticipated minimal energy savings or non-energy-related side effects \parencite{guven, mexico_congress, congress}. However, uncertainty about the overall effects of DST has hindered progress in implementing any changes, e.g in the EU since 2018 \parencite{eu_directive}.
\newline
% The inception of \ac{DST} traces back to the early 20th century when numerous countries embraced this practice as a strategic means to optimise energy usage, particularly during periods of extended daylight. The fundamental premise of \ac{DST} is rooted in the notion that aligning human activities more closely with the natural ebb and flow of daylight can result in a notable reduction in energy consumption for lighting and heating. While the adoption of \ac{DST} for energy conservation purposes has been widespread, the associated environmental implications of this temporal adjustment have become a focal point of debate among scholars and policymakers alike. We propose adding to this discussion and analysing the relationship between \ac{DST} and carbon emissions, and therefore, it's environmental impact.
Our focus on Australia comes from the fact that Australia has heterogeneous adoption of \ac{DST} between states, which independently decide on the policy. After the short-term introduction of DST during both world wars in all states, the policy was reintroduced in Tasmania in 1967 because of a severe drought that led to a power shortage \parencite{Tasmania}.\footnote{The majority of Tasmania's electricity generation comes from hydropower \parencite{aemc_tas}}
In 1971, New South Wales, the Australian Capital Territory, Victoria and Southern Australia also followed \ac{DST}.\footnote{see \cite{NewSouthWales, ACP, Victoria, SouthernAustralia}}\textsuperscript{,}\footnote{Notice that due to its size, the ACT grid is electrically and commercially integrated as part of the NSW grid. Thus, when mentioning NSW this will include the ACT in our study} Since then, \ac{DST} has been active in the south-eastern states, with clocks shifting forward by one hour on the first Sunday of October and backwards on the first Sunday of April. In the Northern Territory, \ac{DST} has not been used after World War II.\footnote{The Northern Territory has a population of roughly 250000 inhabitants \parencite{nt_pop} making \ac{DST} less relevant}
In Western Australia, \ac{DST} was active in different periods, but always quickly discontinued because of strong opposition from different parts of the population, most notably rural inhabitants \parencite{pearce_great_2017}. The main arguments against \ac{DST} are Western Australia´s closer location to the equator, which makes the policy less suitable and its very hot climate - especially in the north-eastern parts - leading to negative effects of \ac{DST} on the rural population in these areas.  
In Queensland, \ac{DST} was active in several short periods, but always heavily discussed. On one hand, the urban coastal regions at the (south) eastern side of Queensland are mostly in favour of \ac{DST} and advances to re-introduce the policy are made every other year \parencite{pearce_history_2017}. On the other hand, most other areas - which have more extreme temperatures -  are opposed to \ac{DST}, arguing that one more hour of sun during the day would lead to very negative effects on their lives \parencite{westcott_daylight_2010}. Furthermore, Queensland is also located closer to the equator, making the state less suitable for the policy. The last referendum on \ac{DST} in Queensland was decided with 55 \% against the policy \parencite{queensland_referendum}, which also illustrates how divided opinions on \ac{DST} are.  
\newline
With respect to total energy generation, Australia's energy grid has a substantial solar capacity (34\%, according to the Australian Energy Regulator (2023)\nocite{state_of_the_market}). The south-eastern states (New South Wales, Victoria, Southern Australia, Tasmania and Queensland) are linked into a single electricity grid, operated as the \ac{NEM} by \ac{AEMO}. 
The breakdown of energy in the \ac{NEM} over the last 12 months was 8\% solar, 63\% coal, 15\% wind, 8\% hydro.
Comparatively, Queensland is close to representative, with 11\%  solar, 74\% coal and 5\% from wind energy \parencite{aemo_fuel_mix}, thus a relatively similar electricity grid.
% I was trying to add some summary stats
% In the end, I don't know how much this really adds
The Australian electricity market is highly skewed.
20\% of annual electricity costs are incurred during the top 1\% of time intervals.\footnote{Own analysis}
Thus a change of ``only" one hour may be surprisingly significant. As shown in Table \ref{tab:sunrise emissions stats}, the generation mix shortly before sunrise, and shortly after sunset have higher emissions intensity than shortly after sunrise and before sunset, with a larger difference in the morning. Thus a shift of load from one time of day to another may not necessarily lead to a zero-sum change.

\subsection{Data}
\label{sec:data}

%\ac{AEMO} publish data for the \ac{NEM} publicly on their website (NEMWeb\nocite{nemweb, nemweb_mmsdm}). I(Alex) rewrote this, see below
Energy and carbon data from 2009 to 2023 was obtained from \ac{AEMO}\nocite{nemweb, nemweb_mmsdm}. 
The dataset does not include Western Australia nor the Northern Territory due to the structure of the \ac{NEM}.
To calculate the emissions per region over time, we take static emissions intensity (tCO2e/MWh) per generator, and join them with energy output (MW) per generator per 5 minutes. We sum to a region level, downsample to half-hourly data, and adjust for inter-region energy flows.\footnote{available only with 30-minute granularity. The calculations are explained in detail in Appendix \ref{sec:interconnector calc}.}
To obtain per capita values, 
quarterly population data per region was downloaded from \citeauthor{abs_population} and linearly interpolated to a daily level.
Weather data was obtained from the Bureau of Meteorology\nocite{weather_data} and Willy Weather\nocite{willy_weather}. Temperature data was taken from capital cities, since most thermal load (building heating and cooling) occurs in capital cities. Wind and sunlight data was taken from region midpoints, as representative values for generation which is typically dispersed across the region. Our final dataset includes information on $CO_2$, $kWh$, temperature, solar exposure, wind, and population, as well as holiday and weekend dates. Table \ref{tab:summary stats} presents summary statistics for all variables, showing that Queensland and the treated regions are relatively similar.  
